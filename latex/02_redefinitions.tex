%%% У даному файлі визначайте всі необхідні вам нові команди TeX
%%% або робіть перевизначення існуючих, наприклад...

% Перевизначення символу порожньої множини та знаків "більше-дорівнює", "менше-дорівнює" на прийняті у нас
\let\oldemptyset\emptyset
\let\emptyset\varnothing
\let\geq\geqslant
\let\leq\leqslant

% Визначення нових математичних команд
\newcommand*{\binsp}[1]{\ensuremath \left\{0, 1\right\}^{#1}}       % {0, 1}^m
\newcommand*{\xor}{\ensuremath \oplus}                              % \xor = (+)
\newcommand*{\GF}[1]{\ensuremath \mathbb F_{#1}}                    % F_n
\newcommand*{\GFgroup}[1]{\ensuremath \mathbb F^{*}_{#1}}           % F^*_n
\newcommand*{\Zring}[1]{\ensuremath \mathbb Z_{#1}}                 % Z_n
\newcommand*{\Zgroup}[1]{\ensuremath \mathbb Z^{*}_{#1}}            % Z^*_n
\newcommand*{\Jset}[1]{\ensuremath \mathbb J_{#1}}                  % J_n
\newcommand*{\Qset}[1]{\ensuremath \mathbb Q_{#1}}                  % Q_n
\newcommand*{\PQset}[1]{\ensuremath \widetilde{\mathbb Q}_{#1}}     % Q~_n
\newcommand*{\cyclic}[1]{\ensuremath \left\langle {#1} \right\rangle}                  % <g>
\newcommand*{\Legendre}[2]{\ensuremath \left( \frac{#1}{#2} \right)}  % символ Лежандра/Якоби
\newcommand*{\compinv}[1]{\ensuremath {#1}^{\left\langle -1 \right\rangle}}  % обратный по композиции

% Інший спосіб визначення математичного оператору
\DeclareMathOperator{\ord}{ord}
\DeclareMathOperator{\lcm}{lcm}
\DeclareMathOperator{\Li}{Li}
\DeclareMathOperator{\Coef}{Coef}
\DeclareMathOperator{\Log}{Log}
\DeclareMathOperator{\Exp}{Exp}
\DeclareMathOperator{\Res}{Res}
\DeclareMathOperator{\charact}{char}
\DeclareMathOperator{\Sym}{Sym}


% команда для коментарів червоним кольором
% !!! Конфлікт пакету color з якимось іншим пакетом, не використовувати
%\newcommand{\todo}[1]{\textcolor{red}{#1}}


%%% ...і таке інше